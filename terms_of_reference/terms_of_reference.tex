\input{preamble.tex}

\begin{document}

  \begin{titlepage}	% начало титульной страницы

    \thispagestyle{empty}
    \begin{center}		% выравнивание по центру
      \textsc{Московский физико-технический институт} \\
      \textsc{(национальный исследовательский университет)} \\
      \textsc{Физтех-школа прикладной математики и информатики} \\[6cm]

      \large Техническое задание курсового проекта \\
      \huge \textsc{Генератор презентаций} \\ 
    \end{center}

    \vfill

    \begin{flushright} % выравнивание по правому краю
      \begin{minipage}{0.45\textwidth} % врезка в половину ширины текста
        \begin{flushleft} % выровнять её содержимое по левому краю

          \large\textbf{Работу выполнили:}\\
          \large Дорин Даниил Б05-002 \\
          \large Киселев Никита Б05-002 \\
          \large Котцова Лидия Б05-002 \\
          \large Никитина Мария Б05-002 \\

          \large \textbf{Научный руководитель:}\\
          \large доцент Гутник С.А.

        \end{flushleft}
      \end{minipage}
    \end{flushright}

    \vfill % заполнить всё доступное ниже пространство

    \begin{center}
      \large Долгопрудный \\
      \large 2022 % вывести дату
    \end{center} % закончить выравнивание по центру

    \pagebreak

  \end{titlepage} % конец титульной страницы
  \newpage

  \hypertarget{intro}{}
  \tableofcontents
  \newpage

  \section{Цель курсового проекта}

  Целью проекта является создание программы, имеющей пользовательский интерфейс 
  и позволяющей генерировать презентацию на основе указанного текстового файла 
  (в упрощенной версии "--- пояснительной записки). 

  \section{Общая идея задачи}

  Перед пользователем стоит задача создания презентации в краткие сроки.
  Программа позволяет сэкономить большое количество времени и автоматизировать указанный процесс.
  С помощью пользовательского интерфейса имеется возможность указания источника
  (печатный ввод или текстовый файл). Затем, в зависимости от программного обеспечения пользователя,
  выбирается способ генерации презентации "--- \LaTeX~/ внутренним интерфейсом приложения.
  На выходе пользователь получает готовую презентацию, каждый слайд которой соответствует 
  некоторой теме, выделенной в текстовом файле с помощью машинного обучения.

  \section{Основные подзадачи}

  В поставленной задаче можно выделить несколько взаимосвязаных подзадач, а именно:
  \begin{itemize}
    \item Поиск данных для обучения программы;
    \item Разработка приятного и понятного пользователю графического интерфейса;
    \item Выделение основных тем с помощью машинного обучения;
    \item Создание презентации с помощью \LaTeX;
    \item Создание презентации <<с нуля>> в случае отсутствия \LaTeX~на устройстве;
    \item Иллюстрирование текста на слайдах. 
  \end{itemize}

  \section{Детальное описание содержания подзадач}

  \begin{itemize}
    \item \textbf{Поиск данных для обучения программы:}
    Найти готовые презентации различной сложности (темы простые и легко различимые;
    темы средней сложности; сложноразличимые темы).
    \item \textbf{Разработка приятного и понятного пользователю графического интерфейса:}
    Необходимо реализовать выбор источника (печатный ввод или текстовый файл) и
    способа создания презентации (\LaTeX~/ внутренним интерфейсом приложения).
    \item \textbf{Выделение основных тем с помощью машинного обучения:}
    Предобработка текстового файла, разделение его на предложения, выделение в нем основных тем
    с последующим разделением всего документа на связные осмысленные абзацы (разделения производятся
    между предложениями с наиболее различными темами).
    \item \textbf{Создание презентации с помощью \LaTeX:}
    Используя разделение текстового файла на абзацы, полученное в предыдущем пункте,
    требуется скомпилировать \LaTeX-презентацию, где один слайд соответсвует одному абзацу разделения.
    \item \textbf{Создание презентации <<с нуля>> в случае отсутствия \LaTeX~на устройстве:}
    Используя разделение текстового файла на абзацы, полученное в третьем пункте,
    требуется посредством создания пустых слайдов и помещения на них текстовых записей
    сгенерировать необходимый PDF-файл. 
    \item \textbf{Иллюстрирование текста на слайдах:}
    Используя ключевые слова каждой темы, полученные в третьем пункте, требуется реализовать
    поиск изображений, иллюстрирующих содержимое каждого слайда.
  \end{itemize}

  \newpage
  \section{Состав работ и исполнители}

  \textbf{Подзадачи} распределяются следующим образом:
  \begin{itemize}
    \item Поиск данных для обучения программы \\ Никитина Мария \\ 24 апреля;
    \item Разработка графического интерфейса \\ Дорин Даниил \\ 8 мая;
    \item Выделение основных тем с помощью машинного обучения \\ Киселев Никита \\ 1 мая;
    \item Создание презентации с помощью \LaTeX \\ Киселев Никита \\ 1 мая;
    \item Создание презентации <<с нуля>> в случае отсутствия \LaTeX~на устройстве \\ Никитина Мария \\ 1 мая;
    \item Иллюстрирование текста на слайдах \\ Котцова Лидия \\ 8 мая.
  \end{itemize}

  \textbf{Дополнительные работы:}
  \begin{itemize}
    \item Разработка технического задания \\ Котцова Лидия \\ 3 апреля;
    \item Свод воедино и тестирование \\ Дорин Даниил \\ 12 мая.
  \end{itemize}

  \section{Используемые программные и технические средства}

  \begin{center}
    Coming soon \ldots
  \end{center}

\end{document}