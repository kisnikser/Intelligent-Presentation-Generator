\begin{frame}
	С точки зрения научной систематики, домашняя кошка — млекопитающее семейства кошачьих отряда хищных. Нередко домашнюю кошку рассматривают как подвид лесной кошки, однако, с точки зрения современной биологической систематики (2017 год), домашняя кошка является отдельным биологическим видом. Являясь одиночным охотником на грызунов и других мелких животных, кошка — социальное животное, использующее для общения широкий диапазон звуковых сигналов, а также феромоны и движения тела. В настоящее время в мире насчитывается около 600 млн домашних кошек, выведено около 200 пород, от длинношёрстных (персидская кошка) до лишённых шерсти (сфинксы), признанных и зарегистрированных различными фелинологическими организациями. На протяжении 10 000 лет кошки ценятся человеком, в том числе за способность охотиться на грызунов и других домашних вредителей. Чай — напиток, получаемый варкой, завариванием и/или настаиванием листа чайного куста, который предварительно подготавливается специальным образом. Чай — также сам лист чайного куста, обработанный и подготовленный для приготовления напитка.
\end{frame}

\begin{frame}
	Подготовка включает предварительную сушку (вяление), скручивание, более или менее длительное ферментативное окисление, окончательную сушку.
\end{frame}

\begin{frame}
	Прочие операции вводятся в процесс только для производства отдельных видов и сортов чая. Иногда слово «чай» используют и в качестве названия чайного куста — вида растений рода Камелия семейства Чайные; в ботанической научной литературе для этого вида обычно используется название камелия китайская. Чай в широком смысле — любой напиток, приготовленный путём заваривания предварительно подготовленного растительного материала. В названиях таких напитков к слову «чай», как правило, добавляется пояснение, характеризующее используемое сырьё («травяной чай», «ягодный чай», «фруктовый чай» и так далее). Ноутбук — переносной компьютер, в корпусе которого объединены типичные компоненты ПК, включая дисплей, клавиатуру и устройство указания (обычно сенсорная панель или тачпад), а также аккумуляторные батареи. Ноутбуки отличаются небольшими размерами и весом, время автономной работы ноутбуков варьируется в пределах от 2 до 15 часов. Лэптоп — более широкий термин, он применяется как к ноутбукам, так и нетбукам, смартбукам. К ноутбукам обычно относят лэптопы (часто употребляется «лаптоп»), выполненные в раскладном форм-факторе. Ноутбук переносят в сложенном виде, это позволяет защитить экран, клавиатуру и тачпад при транспортировке. Также это связано с удобством транспортировки (чаще всего ноутбук транспортируется в портфеле, что позволяет не держать его в руках, а повесить на плечо).
\end{frame}

